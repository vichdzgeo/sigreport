\documentclass[twoside]{article}
\usepackage{supertabular}
\usepackage{array}
\usepackage{multicol}
\usepackage{graphicx}
\usepackage[textwidth=20cm,textheight=24cm]{geometry}
\usepackage{hhline}
\usepackage{fontspec}
\setmainfont{Calibri}
\usepackage[table, dvipsnames]{xcolor}
\usepackage{multirow}

\definecolor{myblue1}{RGB}{141,179, 226}
\definecolor{myblue2}{RGB}{79,129, 189}
\definecolor{myblue3}{RGB}{74,126, 187}
\definecolor{mykhaki}{RGB}{168,177, 148}
\newcommand{\thiscolor}[1]{\texttt{#1} \hfill \fcolorbox{black}{#1}{\hspace{2mm}}}
\setlength{\headheight}{15pt}
\usepackage{fancyhdr}
\pagestyle{fancy}
\fancyhf{} 
\fancyhead[C]{\small \textcolor{mykhaki} {Proyecto Colibrí}}
\fancyhead[LO,LE]{\small \textcolor{mykhaki} {Manifestación de Impacto Ambiental modalidad regional}}
\fancyhead[RE,RO]{\begin{tabular}{cc}\small \textcolor{mykhaki} {Capítulo 2  (Sección II)} & \small \textcolor{mykhaki} { Pág. \thepage}\end{tabular}}
\usepackage{xpatch}
\xpretocmd\headrule{\color{mykhaki}}{}{\PatchFailed}
\begin{document}
\begin{flushleft}
\tablefirsthead{}
\tablehead{}
\tabletail{}
\tablelasttail{}
\arrayrulecolor{myblue3}

\begin{tabular}{m{7.7320004cm}m{0.091000006cm}|m{2.421cm}|m{0.089cm}m{0.694cm}m{0.693cm}m{0.693cm}m{0.693cm}m{0.693cm}m{0.691cm}m{0.693cm}m{0.703cm}}
\hhline{~~-~~~~~~~~~}
 ~&~ & \multirow{2}{*}{\centering\textcolor{myblue2}{Clave: PUEB}}&~ &\footnotesize Fase:&\footnotesize\textcolor{myblue2} {\textbf{2}}&~&~&~&~&~&~
\\\hhline{~~~~~------~}
\textcolor{myblue2} {Componente:}
 &
~
 &
 &
~
 &
\multicolumn{1}{m{0.694cm}|}{\footnotesize Años: } &
\multicolumn{1}{m{0.693cm}|}{\footnotesize\centering 1-5} &
\multicolumn{1}{m{0.693cm}|}{\cellcolor{gray!50}\footnotesize\centering 6-10} &
\multicolumn{1}{m{0.693cm}|}{\footnotesize\centering 11-15} &
\multicolumn{1}{m{0.693cm}|}{\cellcolor{white}\footnotesize\centering 16-20} &
\multicolumn{1}{m{0.691cm}|}{\footnotesize\centering 21-25} &
\multicolumn{1}{m{0.693cm}|}{\footnotesize\centering 26-30} &
~
\\\hhline{~~-~~------~}
\textcolor{myblue2} {Pueblo}
 &
~
 &
\cellcolor{myblue1}\footnotesize\centering\color{white}\textbf{ Construcción}
 &
~
 &

\multicolumn{8}{l}{\footnotesize Duración de la obra o actividad: \textcolor{myblue2} {\textbf{1,457 días en 5 años}}}


\\\hhline{~~-~~~~~~~~~}

\end{tabular}
\end{flushleft}

{\color{myblue2} \rule{\linewidth}{0.6mm} }

\begin{multicols}{2}

\bigskip

\footnotesize\textcolor{myblue2} {\textbf{ Localización}}


\bigskip

 \includegraphics{texto-img/texto-img001.jpg} 


\bigskip


\bigskip

\footnotesize Categorías del POEL: Duna embrionaria (DEM); dunas terciarias/vegetación forestal (DTF); dunas terciarias/cobertura no forestal (DTN); dunas terciarias/vegetación secundaria de selva (DTS); primer cordón de dunas (PCD); selva baja caducifolia (SBC); vegetación forestal (VFO); cobertura no forestal (CNF); vegetación secundaria de selva (VSS).


\bigskip


\bigskip

\textcolor{myblue2} {\textbf{Descripción general del componente para esta fase}}


\bigskip

\footnotesize El componente “Pueblo” se construirá a lo largo de dos fases (años 1 al 10) e integrará edificaciones relacionadas con actividades hoteleras y residenciales, actividades sociales, deportivas, artísticas y culturales, actividades comerciales y de servicios, pesca y venta de productos frescos del mar.

La fase 1 incluirá la construcción de dos hoteles boutique así como múltiples villas residenciales para alojar a visitantes (véase la tabla Construcción de producto inmobiliario para esta fase), familias de pescadores locales, trabajadores y operadores del condominio. Para las actividades comerciales y de servicios se construirán 25 edificaciones consistentes en tres restaurantes, cuatro mercados de misceláneos y productos del mar, siete comercios de productos diversos, una capilla, un módulo de escamoche, almacén y lavado de pescado, dos bodegas de refrigeración, un módulo de baños públicos y bodega, cinco establecimientos de servicios varios tales como un consultorio de atención médica de primer contacto, delegación municipal, una lavandería y dos oficinas. Para las actividades recreativas y culturales, en esta fase se construirá un gimnasio. Se construirá un estacionamiento para facilitar el ingreso a los hoteles boutique y áreas aledañas. 

La totalidad de las edificaciones de este componente se clasificarán en dos sistemas constructivos: palafitos y edificación convencional (ver sistemas constructivos al final de este apartado), y respetarán una altura máxima de 8 m o dos niveles (incluyendo la planta baja y tomando en cuenta la altura a nivel de piso natural) para mantener la visibilidad hacia la costa; de acuerdo a lo que establece el POEL en relación a las edificaciones que se ubiquen a una distancia menor a 200 m de la Zona Federal Marítimo Terrestre.

La ubicación donde se desplantará este componente coincide con la actual localización del punto público de acceso vehicular a la playa, en donde existe un asentamiento humano de familias de pescadores conformado por seis palapas o enramadas, donde viven itinerantemente alrededor de 12 familias (ver Medidas de responsabilidad social). 

Además, en este sitio se encuentra la infraestructura que actualmente alberga las instalaciones del Campamento de Conservación de Tortugas Marinas administrado por la Comisión Nacional de Áreas Naturales Protegidas (CONANP) en Chalacatepec. Dicha infraestructura será remplazada por instalaciones nuevas y diseñada de acuerdo al programa de necesidades apegado a los requerimientos de la normatividad ambiental vigente; para tal efecto, será necesario demoler las actuales (ver Demolición de las instalaciones del actual Campamento Tortuguero) y el Campamento será reubicado con la finalidad de mejorar las condiciones constructivas y operativas de las actuales instalaciones. 

Además de la superficie edificable, como parte de las áreas exteriores del componente se construirán vialidades secundarias y estacionamientos de empedrado asentado en arena (ver Sistema constructivo de vialidades secundarias convencionales); terrazas y explanadas, áreas verdes ornamentales y deportivas, dos albercas y diversos andadores de acceso al mar; todo esto empleando sistemas constructivos convencionales. 

Se construirán las redes secundarias de infraestructura hidráulica, sanitaria y de agua tratada requeridas para garantizar el abastecimiento de agua potable y la conducción del agua residual hacia su proceso de tratamiento, y desde el mismo, de vuelta a su origen para ser reutilizada en el riego de áreas verdes ornamentales. El componente cuenta con su propia Planta de Tratamiento al interior del mismo (ver componente Plantas de Tratamiento de aguas residuales y manejo de aguas residuales) y una línea de descarga de excedentes de agua tratada hasta un estanque artificial.

Se construirán además las obras secundarias de infraestructura eléctrica de baja tensión y de telecomunicaciones. La siguiente tabla muestra la totalidad de las redes de infraestructura a implementarse en esta fase.

\bigskip

\begin{tabular}{|c|c|c|c|}
\hline
\multicolumn{4}{|l|}{\cellcolor{myblue1}\footnotesize\color{white}{ Longitudes de infraestructura a implementar por categoría del POEL (km)}}\\
\hline
\multicolumn{1}{|c|}{\textcolor{myblue2} {Tipo de red}} & \textcolor{myblue2} {VFO} & \textcolor{myblue2} {DTF} & \textcolor{myblue2} {CNF}\\
\hline
\multicolumn{1}{|l|}{Hidráulica} & 1.4 & 0.3 & 0.2\\
\hline
\multicolumn{1}{|l|}{Sanitaria} & 1.4 & 0.4 & 0.2\\
\hline
\multicolumn{1}{|l|}{Agua tratada} & 1.6 & 0.4 & 0.2\\
\hline
\multicolumn{1}{|l|}{Telecomunicaciones} & 1.2 & 0.6 & 0.2\\
\hline
\multicolumn{1}{|l|}{Eléctrica aérea} & 0.2 & 0 & 0.1\\
\hline
\multicolumn{1}{|l|}{Eléctrica subterránea} & 1.0 & 0.5 & 0.3\\
\hline
\end{tabular}

\bigskip

Previo al inicio de los procesos de obra, las actividades del componente incluyen la construcción de edificaciones de carácter provisional denominadas Obras temporales, las cuales se ubicarán en la brecha existente que bordea el componente al límite sureste así como en el sitio donde será la explanada central del componente el cual presenta Cobertura No Forestal; dichos espacios tendrán la finalidad de servir como áreas auxiliares de trabajo y almacenamiento desde donde los trabajadores se desplazarán al lugar de trabajo corres-pondiente. Estas obras no formarán parte del componente pero son necesarias para alber-gar de forma temporal trabajadores, insumos, maquinaria, equipos, etc.; y serán retiradas una vez concluida la totalidad del componente al término de la fase 2.
\end{multicols}

\end{document}
